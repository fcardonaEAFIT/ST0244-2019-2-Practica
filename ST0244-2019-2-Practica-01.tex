% -*- ispell-local-dictionary: "spanish"; -*-

\documentclass{article}

\usepackage[utf8]{inputenc}
\usepackage[spanish]{babel}
\usepackage{amssymb,latexsym}
\usepackage{amsmath}
\usepackage{amsthm}
\usepackage{verbatim}
\usepackage{alltt}
\usepackage{url}
% \usepackage{algorithm}
% \usepackage{algorithmc}
\usepackage[ruled,vlined]{algorithm2e}
\usepackage[%
backend=biber,%
backref=true,%
bibstyle=authoryear,%
citestyle=authoryear-comp,%
giveninits=true,%
maxbibnames=4,%
maxcitenames=8,%
sorting=ynt,%
uniquename=init,%
]{biblatex}
\usepackage{csquotes}
\usepackage{hyperref}

%%%%%%%%%%%%%%%%%%%%%%%%%%%%%%%%%%%%%%%%%%%%%%%%%%%%%%%%%%%%%%%%%%%%%%%%%%%%%%
% Layout

\newcommand{\TI}[1]{\text{\textit{#1}}}
\newcommand{\TC}[1]{\text{'\textbf{#1}'}}
\newcommand{\TL}[1]{\text{\texttt{#1}}}
\newcommand{\TP}[1]{'\mathbf{#1}'}
\newcommand{\UNUMBER}[1]{\begin{math}_{#1}\end{math}}

%%%%%%%%%%%%%%%%%%%%%%%%%%%%%%%%%%%%%%%%%%%%%%%%%%%%%%%%%%%%%%%%%%%%%%%%%%%%%%
% Biblatex stuff

\addbibresource{ST0244-2019-2-Practica-01.bib}

\DefineBibliographyStrings{spanish}{%
  references = {Bibliografía},
}

% Remove quotes from the title.
\DeclareFieldFormat%
  [article,%
  incollection,%
  inproceedings,%
  thesis,% works for mastersthesis and phdthesis
  unpublished,%
  ]{title}{#1}

% Remove the quotes from the title in \citetext. The star version is
% used to redefine citetitle for all entry types. (see
% http://tex.stackexchange.com/questions/122665/suppress-quotes-or-other-markup-in-citetitle)
% \DeclareFieldFormat*{citetitle}{#1}

% Remove "In:" before journal names.
%
% From
% http://tex.stackexchange.com/questions/10682/suppress-in-biblatex
\renewbibmacro{in:}{%
  \ifentrytype{article}{}{\printtext{\bibstring{in}\intitlepunct}}}

% Remove italics from the title names.
\DeclareFieldFormat%
 [book,%
 misc,%
 online,%
 report,%
 ]{title}{#1}

% Remove italics from journal names.
\DeclareFieldFormat{journaltitle}{#1}

% Remove italics from proceedings names in articles in proceedings.
\DeclareFieldFormat%
[incollection,%
inproceedings,%
]{booktitle}{#1}

% Remove markup in \citetitle.
%
% From:
% https://tex.stackexchange.com/questions/122665/suppress-quotes-or-other-markup-in-citetitle/387845.
\DeclareFieldFormat*{citetitle}{#1}

% Remove the url field from some entry types.
% \DeclareFieldFormat[article,thesis]{url}{}

% All authors are printed as "last name – first name".
\DeclareNameAlias{sortname}{family-given}

% All editors are printed as "last name – first name".
\DeclareNameAlias{default}{family-given}

% Using square brackets with authoryear citestyle. From:
% http://tex.stackexchange.com/questions/16765/biblatex-author-year-square-brackets
\makeatletter

\newrobustcmd*{\parentexttrack}[1]{%
  \begingroup
  \blx@blxinit
  \blx@setsfcodes
  \blx@bibopenparen#1\blx@bibcloseparen
  \endgroup}

\AtEveryCite{%
  \let\parentext=\parentexttrack%
  \let\bibopenparen=\bibopenbracket%
  \let\bibcloseparen=\bibclosebracket}

\makeatother

%%%%%%%%%%%%%%%%%%%%%%%%%%%%%%%%%%%%%%%%%%%%%%%%%%%%%%%%%%%%%%%%%%%%%%%%%%%%%%
% Hyperref stuff

\hypersetup{colorlinks=true}

%%%%%%%%%%%%%%%%%%%%%%%%%%%%%%%%%%%%%%%%%%%%%%%%%%%%%%%%%%%%%%%%%%%%%%%%%%%%%%
% Mathematics stuff

\newcommand{\mb}[1]{\mathbf{#1}}
% \newcommand{\mbb}[1]{\mathbb{#1}}
\newcommand{\mc}[1]{{\mathcal{#1}}}
\newcommand{\mf}[1]{\mathfrak{#1}} % Requires \usepackage{amssymb}
\newcommand{\mr}[1]{\mathrm{#1}}
\newcommand{\ms}[1]{\mathsf{#1}}

\theoremstyle{definition}
\newtheorem{example}{Ejemplo}[section]

%%%%%%%%%%%%%%%%%%%%%%%%%%%%%%%%%%%%%%%%%%%%%%%%%%%%%%%%%%%%%%%%%%%%%%%%%%%%%%
\title{Primera práctica de lenguajes de programación\\Algoritmo de unificación\\Implementación en C++}
\date{26 de Septiembre de 2019}
\author{Edwin Andrey Duque Loaiza\\Andrés Sicard Ramírez\\Edison Valencia Díaz\\Juan Francisco Cardona Mc'Cormick}


\begin{document}
\maketitle{}

\section{Introducción}
\label{sec:Introduccion}

En la verificación o inferencia de tipos en los lenguajes de
programación, se requiere comprobar que dos tipos son equivalentes. La
noción de equivalencia dice que dos tipos son equivalentes si se
encuentra un sustitución (ver sección~\ref{sec:estrsust}) que
transforme ambos tipos en el mismo. Por ejemplo, si tenemos el tipo de
dato $\tau_1 = \text{Int}$ y el tipo de dato $\tau_2 = \text{Int}$, la
equivalencia entre ambos es obvia:
\begin{equation*}
  \text{Int} \equiv \text{Int}.
\end{equation*}
Puesto ambos son iguales su sustitución $\theta$ es vacía.

En otros casos, como: $\tau_3 = a \to a$ y $\tau_4 = \text{Int}
\to \text{Int}$, se requiere de un poco más de esfuerzo
\begin{equation*}
  a \to a \equiv \text{Int} \to \text{Int}.
\end{equation*}
En esta ecuación se debe sustituir uno de los lados de la ecuación
para obtener algo que sea equivalente, en este caso la sustitución
obvia es sustituir a la variable $a$ por el tipo $\text{Int}$ y
aplicar dicha sustitución ($\theta = a \mapsto \text{Int}$) a la
ecuación, lo que nos produce:
\begin{equation*}
  \text{Int} \to \text{Int} \equiv \text{Int} \to \text{Int}.
\end{equation*}

\textcite{robinson-1971} definió un algoritmo llamado de unificación
que permite solucionar este tipo de problemas. En los cuales recibe
una serie de restricciones de la forma $T \equiv S$ y encuentra un
conjunto de sustituciones que permite transformar un tipo en el
otro. El algoritmo de unificación para los tipos de datos sirve para
``igualar'' dos tipos (ver sección~\ref{sec:lenguajetipos}).

\section{Algoritmo}\label{sec:algoritmo}

El algoritmo~\ref{alg:unificacion} es tomado de
\textcite{pierce-2002}.

\begin{algorithm}[h!]
  \DontPrintSemicolon
  \LinesNumbered
  \KwData{$C = \{(S \equiv T) \mid S \in \text{Type} \wedge T \in \text{Type} \}$ }
  \KwResult{$\Theta = \{(X \mapsto T) \mid X \in Var \wedge T \in \text{Type}\}$ }
  \Begin{
    \If {$C = \emptyset$}{
      $[\ ]$\;
    }
    \Else{
      \textbf{let} $\{S \equiv T\} \cup C' = C$ \textbf{in}
      \If{$S = T$}{
        $\ms{unify}(C')$;
      }
      \ElseIf{$S = X\ \text{\textbf{and}}\  X \notin FV(T)$}{
        $\ms{unify}([X \mapsto T]C') \circ [X \mapsto T]$;
      }
      \ElseIf{$T = X\ \text{\textbf{and}}\  X \notin FV(T)$}{
        $\ms{unify}([X \mapsto S]C') \circ [X \mapsto S]$;
      }
      \ElseIf{$S \equiv S_1 \to S_2\ \text{\textbf{and}}\ T \equiv T_1 \to T_2$}{
        $\ms{unify}(C' \cup \{S_1 \equiv T_1, S_2 \equiv T_2\})$;
      }
      \Else{
        \emph{fail};
      }
    }
  }
  \caption{$\ms{unify}(C)$}\label{alg:unificacion}
\end{algorithm}

El algoritmo toma como entrada un conjunto de restricciones $C$, donde
cada restricción es un par ordenado de la forma $\{ S = T \}$ y cada
elemento del par ordenado es un tipo (ver
sección~\ref{sec:lenguajetipos}). El algoritmo retorna un conjunto de
sustituciones $\Theta$ y cada sustitución es de la forma
$\{ X \mapsto T\}$, donde $X \in Var$ es una variable y $T$ es un tipo
(ver sección~\ref{sec:lenguajetipos}).

En la línea~2, el algoritmo verifica si el conjunto de
restricciones~$C$ es vacío y retorna un conjunto~$\Theta$ también
vacío.

En la línea~5, el algoritmo separa el conjunto de restricciones~$C$ en
dos partes: la primera es el primer elemento del conjunto
$\{ S \equiv T \}$ y la segunda parte es el resto del
conjunto~$C'$. También en ésta línea se verifica si los tipos básicos
son idénticos, si este es el caso realiza la unificación al resto del
conjunto~$C'$.

En la línea~7, el algoritmo verifica si uno de los tipos es una
variable $X$ y ella no pertence al conjunto de variables libres
$FV(S)$ (ver sección~\ref{sec:fv}) del otro tipo $S$. Si este es el
caso, en la línea~8 realiza la unificación con la aplicación (ver
sección~\ref{sec:aplsus}) de la sustitución $\theta = X \mapsto S$ al
conjunto $C'$; el resultado de la unificación es compuesto~$\circ$
(ver sección~\ref{sec:compsus}) con la sustitución
$\theta = X \mapsto S$.

En la línea~9, el algoritmo verifica si uno de los tipos es una
variable $X$ y ella no pertence al conjunto de variables libres
$FV(T)$ (ver sección~\ref{sec:fv}) del otro tipo $t$. Si este es el
caso, en la línea~8 realiza la unificación con la aplicación (ver
sección~\ref{sec:aplsus}) de la sustitución $\theta = X \mapsto t$ al
conjunto $C'$; el resultado de la unificación es compuesto~$\circ$
(ver sección~\ref{sec:compsus}) con la sustitución
$\theta = X \mapsto T$.

En la línea~11, el algoritmo verifica si ambos tipos son funciones:
$\{S \equiv S_1 \to S_2\}$ y $\{T \equiv T_1 \to T_2\}$. Realiza la
unificación con el conjunto $C'$ unido con dos restricciones más: el
tipo $S_1 \equiv T_1$ y el tipo $S_2 \equiv T_2$.

En la línea~14, el algoritmo llega si ninguna de la anteriores
condiciones se cumple, por lo tanto no puede unificar y falla.

\section{Lenguajes de tipos}\label{sec:lenguajetipos}

Uno de los elementos que requiere el algoritmo de unificación
son los tipos de datos $\tau$:

\[
  \begin{array}[h]{rcl}
    \tau & \to_1  & X \in Var \\
         & \mid_2 & \text{Int} \\
         & \mid_3 & \text{Bool} \\
         & \mid_4 & \tau_1 \to \tau_2 \\
  \end{array}
\]

En la primera producción un tipo es una variable $X$ que pertenece al
conjunto de variables $Var$ (ver sección~\ref{sec:var}). La segunda
producción indica que el tipo es un entero $\text{Int}$. La tercera
producción indica que es un tipo booleano $\text{Bool}$. La cuarta y
última, indica que el tipo es una función que recibe un tipo $\tau_1$
y retorna un tipo $\tau_2$.

\subsection{Ejemplos de tipos}
\label{sec:ejemtipos}

Los tipos básicos son: enteros $\text{Int}$, booleanos $\text{Bool}$ y
variables de tipos $Var$. Ejemplos de los tipos simples:

\[
  \begin{array}[h]{cl}
    \text{Int} & \text{Un entero} \\
    \text{Bool} & \text{Un booleano} \\
    \text{a} & \text{Una variable} \\
    \text{bcd1} & \text{Una variable} \\
  \end{array}
\]

Los tipos funciones permite construir tipos a partir de los más
sencillos:

\[
  \begin{array}[h]{cl}
    \text{Int} \to \text{Int} & \text{Función que toma un entero y retorna un entero} \\
    \text{Int} \to \text{Int} \to \text{Int} & \text{Un operador binario:} +\ -\ \times\ \\
    \text{Int} \to \text{Bool} & \text{Función que toma un entero y verifica una propiedad en él} \\
    \text{a} \to \text{Int} & \text{Una función polimorfica que convierte a entero} \\
    \text{a} \to \text{a} \to \text{a} & \text{Un operador binario genérico} \\
  \end{array}
\]

\section{Estructura de datos}\label{sec:estructdatos}

A continuación se muestra una descripción de los tipos de datos
utilizados por la función $\ms{unify}$ y sus funciones auxiliares (ver
sección~\ref{sec:funaux}).

\subsection{Variables}
\label{sec:var}

La variables son cualquier cadena de carácteres que comienza con una letra,
seguida de una secuencia posiblemente vacía de letras y números. Su expresión
regular:
\begin{equation*}
  [a..zA..Z]([a..zA..Z] \mid [0 ..\ 9])^*
\end{equation*}

\subsection{Tipos}
\label{sec:estrutipos}

Los tipos definidos en la sección~\ref{sec:lenguajetipos} deben ser
definidos como una estructura de datos. Esta estructura debe ser capaz
de representar cualquiera de los cuatro elementos que consta un tipo
$\tau$.

\subsection{Restricción}
\label{sec:estrurest}

Una restricción es un par ordenado de tipos $(S,T)$ donde $T \in \tau$
y $S \in \tau$.

\subsection{Sustitución}
\label{sec:estrsust}

Una sustitución es un par ordenado de tipos $(X,T)$ donde $X \in Var$
y $T \in \tau$. O también una sustitución puede ser definida como una
función
\begin{equation*}
  \theta : \tau \to \tau.
\end{equation*}
Es decir, una sustitución es una función que recibe un tipo y lo
transforma en otro tipo.

\subsection{Conjuntos}
\label{sec:conjuntos}

Existen dos tipos de conjuntos: uno de restricciones~$C$ y otro de
sustituciones~$\Theta$.


\section{Funciones auxiliares}
\label{sec:funaux}

Existe una serie de funciones auxilares que son necesarias en el
algoritmo de $\ms{unify}$.

\subsection{Variables libres, \emph{Free Variables} ($FV$) }
\label{sec:fv}

La función $FV$ recibe un tipo de dato y retorna un conjunto de
variables que son están definidas dentro del tipo:

\[
  FV(T) =
  \begin{cases}
    \{X\}, & \text{si $T = X$ donde $X$ es una variable} \\
    \emptyset & \text{si $T = \text{Int}$} \\
    \emptyset & \text{si $T = \text{Int}$} \\
    FV(\tau_1) \cup FV(\tau_2) & \text{si $T = \tau_1 \to \tau_2$} \\
  \end{cases}
\]

\subsection{Aplicación de la sustitución}
\label{sec:aplsus}

La aplicación $\alpha$ de una sustitución es una operación que está
divida en varias sub-operaciones: en primer lugar la aplicación de la
sustitución en un conjunto de restricciones:
\begin{equation*}
  \alpha :: \theta \to C \to C.
\end{equation*}
Toma una sustitución se la aplica a cada elemento del conjunto de
restricciones y produce un nuevo conjunto de restricciones. Esta
sustitución $\alpha$ está definida de la siguiente forma:
\begin{equation*}
  \alpha =
  \begin{cases}
    \emptyset, & C = \emptyset;
    \\
    \{\alpha\  \theta\  c\} \cup\ \alpha\ \theta\ C', & C = \{c\} \cup
    C'.
  \end{cases}
\end{equation*}
Esta aplicación está basada en otra aplicación de sustitución sobre
una restricción $\rho = \{ S \equiv T \}$, donde $S \in \tau$ y $T \in
\tau$:
\begin{equation*}
  \alpha :: \theta \to \rho \to \rho,
\end{equation*}
y está definida como:
\begin{equation*}
  \alpha = \{(\alpha\ \theta\ S) \equiv (\alpha\ \theta\ T)\}.
\end{equation*}
Esta aplicación también está basada en otra aplicación $\alpha$ pero
sobre tipos sobre los tipos:
\begin{equation*}
  \alpha = \theta \to \tau \to \tau.
\end{equation*}
Si la sustitución~$\theta$ es definida como $[ X' \mapsto S']$,
entonces la aplicación $\alpha$ está definida como:
\begin{equation*}
  \alpha\ \theta\ \tau =
  \begin{cases}
    X, & \text{si $\tau = X$ y $X \neq X'$};
    \\
    S', & \text{si $\tau = X$ y $X = X'$};
    \\
    \text{Int}, & \text{si $\tau = \text{Int}$};
    \\
    \text{Bool}, & \text{si $\tau = \text{Bool}$}
    ;\\
    (\alpha\ \theta\ \tau_1) \to (\alpha\ \theta\ \tau_2), & \text{si
      $\tau = \tau_1 \to \tau_2$}.
  \end{cases}
\end{equation*}

\subsection{Composición de sustituciones}
\label{sec:compsus}

La composición ($\circ$) de sustituciones se puede definir de dos
formas. Primero, si la sustitución está definida como un par ordenado
la composición es una función que toma un conjunto de sustituciones
$\Theta$ y una sustitución $\theta$ y adiciona está última al conjunto
de sustituciones.

Segundo, si la sustitucion está definida como una
función, se puede utilizando el operador de composición de funciones,
que toma dos sustituciones y produce una nueva.

\section{Gramática de entrada}
\label{sec:gramatica-de-entrada}

La siguiente es la definición de la gramática de entrada:

\[
  \begin{array}[h]{rcl}
    \text{Entrada} & \to  & \text{ListaEquiv} \texttt{EOF}
    \\
    \\
    \text{ListaEquiv} & \to & \text{Tipo} = \text{Tipo}\ \texttt{EOL}\
                              \text{ListaEquiv}
    \\
                   & \to & \epsilon
    \\
    \\
    \text{Tipo}    & \to & \text{VAR}\  \text{TipoRest}
    \\
                   & \mid & \text{int}\ \text{TipoRest}
    \\
                   & \mid & \text{bool}\ \text{TipoRest}
    \\
                   & \mid & (\ \text{Tipo} \ )
    \\
    \\
    \text{TipoRest} & \to & \text{-$>$}\ \text{Tipo}
    \\
                   & \mid & \epsilon
  \end{array}
\]

donde \texttt{EOF} es final de fichero y \texttt{EOL} es el carácter
de línea nueva\footnote{En la plataforma cygwin-Windows equivale al
  caracter cuyo valor numérico es~10.}.

\begin{example}
Una entrada correcta es la siguiente:

\begin{alltt}
a = a <EOL>
int = int <EOL>
a -> a = int -> int <EOL>
(a -> a) -> b = (int -> int) -> int <EOL>
<EOF>
\end{alltt}

La entrada muestra un programa de~4 líneas de tipos que será
verificados si son equivalentes o no. Los símbolos \texttt{EOL} y
\texttt{EOF} se refieren a la línea nueva y al final del fichero, el
primer caracter es visible y el segundo es simple un evento.
\end{example}

\section{Requerimientos}
\label{sec:requerimientos}


\subsection{Técnicos}
\label{sec:reqtec}

\begin{description}
\item[Lenguajes de programación:] C++
\item[Nombre ejecutable:] cppunify
\item[Control de versiones:] Se va utilizar el manejador de versiones
  \emph{Subversion}
  \parencite{collins-sussman--fitzpatrick-pilato-2011}.
\item[Repositorio de control de versiones:] Riouxsvn (\url{http://www.riouxsvn.com}).
\item[Estructura de directorios:] La siguiente es la estructura de
  directorios o carpetas o \emph{folders} que tendrá el repositorio,
  donde \texttt{+} es un directorio y \texttt{-} es un fichero.
\begin{alltt}
\end{alltt}

El fichero \texttt{miembros.xml} contiene la definición de los
miembros del grupo. El siguiente es el contenido de un posible grupo.

  {\footnotesize
\begin{alltt}
<grupo>
   <nombre>UnificadorSupremo</nombre>
   <url>https://subversion.assembla.com/svn/unificador/</url>
   <miembro>
      <nombre>Juan Francisco Cardona McCormick</nombre>
      <codigo>198610250010</codigo>
      <email>fcardona@eafit.edu.co</email>
   </miembro>
   <miembro>
      <nombre>Santiago</nombre>
      <codigo>201010001010</codigo>
      <email>xxx@eafit.edu.co</email>
   </miembro>
</grupo>
\end{alltt}
  }

  El \emph{\texttt{url}} identifica el repositorio donde está
  almacenada la práctica.  El nombre del grupo es de una sola palabra,
  en minúsculas. Se debe añadir el correo electrónico en la segunda
  entrega. \emph{\texttt{mainclases}} es una lista con las clases
  principales del proyecto tanto para ANTRL como para CUP.

  El \texttt{$<$NombreGrupo$>$} es el nombre elegido para su grupo un identificador que
  sigue el mismo formato de los identificadores en Java.

  El fichero \texttt{README} contiene la información del contenido de
  las sub-capertas y como compilar cada proyecto.

  Los directorios \texttt{src-[leng]$_i$} son los directorios que
  contienen los fuentes de cada uno de los lenguajes en que han hecho
  las versiones de la práctica:
  \begin{equation*}
    \text{[leng]} \to \text{cplusplus} \mid \text{csharp} \mid \text{clisp} \mid \text{haskell}
  \end{equation*}

  El directorio \texttt{examples} contiene los diferentes ejemplos con
  los cuales corren la práctica.

\end{description}

\subsection{Administrativos}
\label{sec:admin}

\begin{description}
\item[Grupo de trabajo:] Grupo máximo de dos estudiantes. \emph{Cada}
  miembro del equipo debe registrarse en assembla
  (\url{http://www.assembla.com}) con un usuario propio. Uno de los
  miembros debe crear un repositorio donde los miembros del equipo son
  los dueños del repositorio. Deben enviar una invitación al monitor
  de la materia y al profesor de la misma.

\item[Primera entrega:] Viernes 26 de Octubre a las 12:00 m. Esta se
  hará automática a través del sistema de control de versiones. Se
  \emph{debe} cumplir todos los requerimientos señalados en este
  documento. Recuerde que la idea es entregar una gramática en el
  fichero de las gramáticas.

\item[Control de versiones:] El control de versiones no es solamente
  un herramienta que facilite la comunicación entre los miembros del
  grupo y del control de versiones, sino que también ayudará al
  profesor a llevar un control sobre el desarrollo de la práctica. Se
  espera que las diferentes registros dentro del control de versiones
  sean cambios graduales. En caso contrario, se procederá a realizar
  un escutrinio a fondo del manejo de control de versiones para evitar
  fraudes.
\end{description}

%%%%%%%%%%%%%%%%%%%%%%%%%%%%%%%%%%%%%%%%%%%%%%%%%%%%%%%%%%%%%%%%%%%%%%%%%%%%%%
% References

\newrefcontext[sorting=nyt]
\printbibliography[heading=bibintoc]

\end{document}

%%% Local Variables:
%%% mode: latex
%%% TeX-master: t
%%% End:
